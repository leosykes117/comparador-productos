\section{Introducción}
En el presente trabajo se desarrolla la estructuración de un sistema de software 
enfocado en la búsqueda de artículos en la web, con el fin de compararlos y 
mostrarlos, de acuerdo a parámetros establecidos por el usuario en la búsqueda. 
Básicamente, un motor de búsqueda de Internet, que también es conocido como 
“buscador” es una herramienta fundamental para que el usuario, a través de un 
navegador web, sea capaz de encontrar la información o sitio web que está 
buscando dentro de la intrincada telaraña que es Internet. 
Para lo cual se utilizan palabras calves, también llamada “keywords”, las cuales 
son palabras que ingresamos al buscador y este nos devuelve un listado de 
resultados de las paginas web que almacenan información, incluyendo audio, 
video, imágenes, y textos. 
Este buscador de internet funciona básicamente comparando las palabras clave o 
keywords que el usuario ingresa en la caja de búsquedas del mismo, con una 
base de datos en donde se encuentran las direcciones de las páginas web que 
contiene las palabras clave que el usuario solicitó en la búsqueda. El resultado de 
esta es la lista de sitios web en los que se mencionan temas relacionados con las 
palabras clave buscadas. 
Para obtener la información que el usuario busca se utiliza la araña web, que es 
un programa desarrollado con el propósito de analizar las páginas de la WWW 
metódicamente para realizar una copia de las páginas que ha visitado, para luego 
ser procesadas e indexadas para ser utilizadas por los buscadores. El trabajo de 
las arañas web comienza cuando visitan un sitio web de una lista previamente 
definida. Una vez accedido el sitio, la araña identifica los enlaces que esta 
contiene y los añade a otra lista de URLs, las cuales visitara posteriormente de 
acuerdo a una serie de parámetros también previamente definidos. 
En otras palabras, la araña web visita un sitio, descarga sus enlaces a una lista, 
termina con la página inicial y comienza a descargar los enlaces de las páginas 
que ha descargado. Esto se repite sucesivamente para todas las páginas que 
descarga y analiza. 
Finalmente encontrada la información, se realiza un programa el cual es 
encargado de recibir la petición de búsqueda, comparar la misma con las entradas 
en el índice, y ser entregado al usuario como una lista de artículos relacionados a 
su búsqueda y filtros de elección.